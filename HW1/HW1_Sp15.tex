\documentclass[12pt]{article}
\usepackage[left=1in, right=1in, top=1in, bottom=1in, nohead]{geometry} % see geometry.pdf on how to lay out the page. There's lots.
\geometry{letterpaper} % or letter or a5paper or ... etc
% \geometry{landscape} % rotated page geometry
\usepackage{outline}
\usepackage{amsmath}
% See the ``Article customise'' template for come common customisations
\usepackage{graphicx}
\usepackage{epstopdf}
\usepackage{fancyhdr}
\usepackage{hyperref}
\usepackage{setspace}
\usepackage[labelfont=bf]{caption}

\setlength{\headheight}{8pt}

\def\dbar{{\mathchar'26\mkern-12mu d}} 

\pagestyle{fancy}
\fancyhf{}
\renewcommand{\headrulewidth}{0.5pt}
\renewcommand{\footrulewidth}{0.5pt}
\lfoot{\today}
\cfoot{\copyright\ 2015 W.\ F.\ Schneider}
\rfoot{\thepage}
\lhead{\bf{Homework 1}\\ }
\chead{\bf{ND CBE 50557} \\ \bf{Spring 2015}}
\rhead{\bf{Due January 20, 2015}\\ }

\title{University of Notre Dame\\Advanced Chemical Engineering Thermodynamics\\(CBE 60553)}
\author{Prof. William F.\ Schneider}
%\date{} % delete this line to display the current date


%%% BEGIN DOCUMENT
\begin{document}
\ \newline
\begin{center}
\begin{tabular}{|p{6in}|}
\hline
\noindent {\bf Lecture 0: The Context of Computational Chemistry} \\
Find an article in the current literature that uses quantum chemical
calculations in some way to address some chemical or materials problem.  \\ \\
Good potential sources include {\em
  J.\ Am.\ Chem.\ Soc.}, {\em J.\ Phys.\ Chem.}, {\em Science}, {\em
  Angew.\ Chemie}, {\em Phys.\ Chem.\ Chem.\ Phys.}, {\em Inorg.\ Chem.},
{\em Phys. Rev.}, {\em Nature}, {\em J.\ Catal.}, .... \\
\hline
\end{tabular}
\end{center}

\begin{enumerate}
\item Give the full citation of your article, including authors,
  title, journal, year, volume, pages, and doi (digital object identifier).
\item In just two to three sentences, what question do the authors try
  to answer using their calculations?
\item In just two to three sentences, what answer do they arrive at?
\end{enumerate}
\end{document}
